%! Author = Jander Moreira
%! Email =  moreira.jander@gmail.com

\documentclass[a4paper, 11pt]{article}
\usepackage[T1]{fontenc}

\usepackage{amsfonts}
\usepackage{amsmath}
\usepackage{textcomp}
\usepackage[all]{nowidow}
\usepackage{xcolor}
\usepackage{tikz}

\usepackage{enumitem}
\setlist{nosep}

\usepackage{versionchanges}


%% Layout

% geometry
\usepackage{geometry}
\geometry{top = 2.5cm, bottom = 2cm, right = 2.5cm, left = 4cm}

% hyperref
\usepackage{hyperref}
\hypersetup{
    colorlinks,
    urlcolor = blue!20!black,
    linkcolor = blue!10!black,
    citecolor = black!80,
}

% cleveref
\usepackage{cleveref}

% makeidx
\usepackage{makeidx}
\makeindex

% minted
\usepackage[outputdir = ./out]{minted}
% \usemintedstyle{borland}
\newminted{latex}{autogobble, breaklines, bgcolor = blue!5, fontsize = \footnotesize}
\newmintinline{latex}{}
%
% tcolorbox
\usepackage{tcolorbox}
\usepackage{color}
\usepackage{comment}
\tcbuselibrary{skins, listings, minted, breakable}
\tcbset{
    colback = blue!3,
    sharp corners,
    box align = top,
    boxrule = 0pt,
    fontupper = \footnotesize,
    fontlower = \footnotesize,
    minted options={
        fontsize = \footnotesize,
        breaklines,
        autogobble,
    },
    listing engine = minted,
}

%% Text support

% macro arguments formats
\colorlet{argumentcolor}{orange!50!black}
\NewDocumentCommand{\Argument}{ m }{%
    \textcolor{argumentcolor}{$\langle$\normalfont\small\textsl{#1}$\rangle$}%
}
\NewDocumentCommand{\MArg}{ m }{\mbox{\texttt{\{}\Argument{#1}\texttt{\}}}}
\NewDocumentCommand{\OArg}{ m }{\mbox{\texttt{[}\Argument{#1}\texttt{]}}}
\NewDocumentCommand{\LArg}{ m }{\mbox{\texttt{<}\Argument{#1}\texttt{>}}}
\NewDocumentCommand{\PackageName}{ m }{\mbox{\textsf{#1}}}
\NewDocumentCommand{\Deprecated}{}{\textcolor{red!80!black}{(deprecated)}}
\NewDocumentCommand{\FromPackage}{ m }{%
    \tikz\node[draw, rounded corners = 1.5pt, inner sep = 1.5pt,
        font = \sffamily\tiny] {#1};%
}
\NewDocumentCommand{\Empty}{}{%
    \mbox{\normalfont\textcolor{black!60}{\textsl{--empty--}}}%
}
\NewDocumentCommand{\Option}{ m }{%
    \mbox{\textcolor{green!40!black}{\texttt{#1}}}%
}
\NewDocumentCommand{\OptionInd}{ m }{%
    \index{#1@\texttt{#1}}%
    \Option{#1}%
}
\NewDocumentCommand{\OptionRef}{ m }{%
    \hyperref[option:#1]{\Option{#1}}%
}
\NewDocumentCommand{\Macro}{ m }{%
    \expandafter\latexinline\expandafter{\csname#1\endcsname}%
}
\NewDocumentCommand{\MacroRef}{ m }{%
    \hyperref[macro:#1]{\Macro{#1}}%
}
\NewDocumentCommand{\MacroDef}{ m }{%
    \index{#1@\texttt{\textbackslash #1}}%
    \refstepcounter{MacroCounter}%
    \label{macro:#1}%
    \Macro{#1}%
}
\NewDocumentCommand{\MacroRefInd}{ m }{%
    \index{#1@\texttt{\textbackslash #1}}%
    \MacroRef{#1}%
}

\tcbset{
    description/.style = {
        coltitle = black,
        fontupper = \normalsize,
        colbacktitle = white,
        titlerule = 0.001pt,
        enhanced jigsaw,
        breakable,
        width = \dimexpr \linewidth - 2em \relax,
        flush right,
        top = 0.5ex,
        bottom = 0pt,
        left = 0pt,
        right = 0pt,
        opacitybacktitle = 0,
        opacityframe = 0,
        opacityback = 0,
    }
}

\NewDocumentEnvironment{macro}{ m O{} o }{%
    %! formatter = off
    \index{#1@\texttt{\textbackslash#1}}%
    \refstepcounter{MacroCounter}%
    \label{macro:#1}%
%! parser = off
    \IfValueTF{#3}{%
        \begin{macro*}{#1}{#2}
    }{%
        \begin{macro*}{#1}{#2}[#3]%
    }
    %! parser = on
        }{%
%! parser = off
    \end{macro*}
    %! parser = on
    %! formatter = on
}
\newcounter{MacroCounter}
\NewDocumentEnvironment{macro*}{ m m o }{
    \medskip\par%
    \begin{tcolorbox}[
        title = {\hspace{-2em}\Macro{#1}#2\IfValueT{#3}{\latexinline!{#3} !}},
        description,
    ]
    }{
    \end{tcolorbox}%
    \medskip%
}

\newlength{\docassignment}
\NewDocumentEnvironment{option}{ m m o }{%
    \label{option:#1}%
    \settowidth{\docassignment}{#2}%
    \begin{tcolorbox}
        [
        title = {%
            \hspace{-2em}\OptionInd{#1}%
            \ifdim\docassignment>0pt\Option{ = #2}\fi%
            \IfValueT{#3}{\hfill\textit{Default:} \Option{#3}}
        },
        description,
        ]
            }{
    \end{tcolorbox}%
    \medskip%
}
\NewDocumentEnvironment{option*}{ m }{%
    \begin{tcolorbox}[title = {\hspace{-2em}#1}, description]
    }{
    \end{tcolorbox}%
    \medskip%
}
\NewDocumentEnvironment{optionnoind}{ m m }{%
    \begin{tcolorbox}[
        title = {\hspace{-2em}\Option{#1 = #2}},
        description,
    ]
    }{
    \end{tcolorbox}%
    \medskip%
}

%% Repetitive text
\NewDocumentCommand{\MacroOptionsText}{}{%
    Any \Argument{options} specified uniquely affect this macro.%
}
\NewDocumentCommand{\BlockOptionsText}{}{%
    Any of the \Argument{options} specified in this macro will affect this command and all items in the inner block, propagating up to and including the closing macro.%
}

%%%%%%%%%%%%%%%%%%%%%%%%%%%%%%%%%%%%%%%%%%%%%%%%%%%%%%%%%%%%%%%%%%%%%%

\title{%
    The \PackageName{versionchanges} package\thanks{This document corresponds to \PackageName{versionchanges}~v\VCVersion, dated \VCDate.
    This text was last revised \today.}%
}
\author{Jander Moreira\\\texttt{moreira.jander@gmail.com}}
\date{\VCDate}

%%%%%%%%%%%%%%%%%%%%%%%%%%%%%%%%%%%%%%%%%%%%%%%%%%%%%%%%%%%%%%%%%%%%%%


\begin{document}
\maketitle
\sloppy

\begin{abstract}
    The \PackageName{versionchanges} package helps to keep a record of the changes in a versioned document.
\end{abstract}

\tableofcontents

% actual versions
\VCRegisterVersion{0.1}{2024/05/03}

% fake versions for examples
\VCRegisterVersion{example}{1800/01/01}
\VCRegisterVersion{example-a}{1802/01/01}

\VCPrintChanges
\textit{Warning}: \textbf{Change History} includes the examples also.


\section{Introduction}

\VCChange[disable]{
    type = released,
    version = 0.1,
    description = {Initial version}
}
This is ``yet another'' package for tracking documentation changes. There is not much new in relation to other packages for manually marking changes, such as \PackageName{changes}\footnote{\url{https://ctan.org/pkg/changes}.}, \PackageName{marginnote}\footnote{\url{https://ctan.org/pkg/marginnote}} and \PackageName{todonotes}\footnote{\url{https://ctan.org/pkg/todonotes}.}, the latter used as support for presenting margin notes.

This package, in particular, is tailored to record versions and types of changes, presenting the change as a note in the margin of the text and generating a list of changes for each version created.


\section{Package usage}

This package depends on the following packages:

\begin{center}
    \begin{tabular}{ll}
        \PackageName{array}      & \url{https://ctan.org/pkg/array} \\
        \PackageName{marginnote} & \url{https://ctan.org/pkg/marginnote} \\
        \PackageName{multicol}   & \url{https://ctan.org/pkg/multicol} \\
        \PackageName{pgfkeys}    & \url{https://www.ctan.org/pkg/pgfkeys} \\
        \PackageName{ragged2e}   & \url{https://www.ctan.org/pkg/ragged2e} \\
        \PackageName{todonotes}  & \url{https://www.ctan.org/pkg/todonotes} \\
    \end{tabular}
\end{center}

To use the package, simply request its use in the preamble of the document.

\begin{macro*}{usepackage}{}[versionchanges]
    There are no package options available at this time.
\end{macro*}


\section{Registering version and adding changes}

The first step is to register each version. This is done through \Macro{RegisterVersion}, informing the version and its date.

\begin{macro}{VCRegisterVersion}[\MArg{version}\MArg{date}]
    The argument \MArg{version} is the version number, such as 1.0, 3.2-beta and so on, and \MArg{date} is intended to registers the date of that version. Actually this date can be any text.
\end{macro}

\begin{latexcode}
    \VCRegisterVersion{example}{1800/01/01}
    \VCRegisterVersion{example-a}{1802/01/01}
\end{latexcode}

Any new change is registered with \Macro{VCChange}.

\begin{macro}{VCChange}[\OArg{todonotes options}\MArg{change info}\OArg{header}]
    This macro register a single change. The change info is a comma separated list with key/value pairs:
    \begin{itemize}
        \item \OptionInd{type}: the type of the change, which can be \OptionInd{released}, \OptionInd{new}, \OptionInd{updated}, \OptionInd{removed}, and \OptionInd{deprecated};
        \item \OptionInd{version}: one of the registered versions (\MacroRef{VCRegisterVersion});
        \item \OptionInd{description}: the text to the list of changes
    \end{itemize}

    \medskip
    Among the change info, the key \OptionInd{suppress} can be specified, causing the item to be supressed from the listing.

    The optional \Argument{header} is a text to be used as a header in the margin note.

    A first optional argument, \Argument{todonotes options}, can be used to pass any adicional options that apply to a \PackageName{todonotes}'s \Macro{todo}. For example, if \Option{disable} is used, the margin note is hidden.
\end{macro}

\begin{tcblisting}{}
    % Update example
    \VCChange{type = updated, version = example, description = {All examples were replaced.}}These are all the examples for this package:\ldots
\end{tcblisting}

\begin{tcblisting}{}
    % New example
    \VCChange{type = new, version = example, description = {Added two-colum support.}}[Two-column]A package option were included to handle two-column documents\ldots
\end{tcblisting}

\begin{tcblisting}{}
    % Hidden update which shows only in the history of changes
    \VCChange[disable]{type = updated, version = example-a, description = {Floating environments are no long supported.}}The use in floats is complex and will not be supported.\ldots
\end{tcblisting}

\begin{macro}{VCPrintChanges}
    The \Macro{VCPrintChanges} creates an unnumbered section called \textit{Change History} and presents the registered changes.

    The order of the versions is defined by the order of creation with \MacroRef{VCRegisterVersion}. The individual changes are in order of appearance in the document.

    A two-pass compilation is needed to update the listing.
\end{macro}

\begin{tcblisting}{}
    \VCPrintChanges
\end{tcblisting}

\begin{tcblisting}{}
    \begingroup
    \VCSet{
        version prefix = {Version:~},
        version style = \color{blue}\small,
        change style = \color{red}\footnotesize,
    }
    \VCPrintChanges
    \endgroup
\end{tcblisting}

\printindex


\end{document}
%%%%%%%%%%%%%%%%%%%%%%%%%%%%%%%%%%%%%%%%%%%%%%%%%%%%%%%%%%%%%%%%%%%%%%